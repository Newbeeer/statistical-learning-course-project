%
% This is the LaTeX template file for lecture notes for CS294-8,
% Computational Biology for Computer Scientists.  When preparing 
% LaTeX notes for this class, please use this template.
%
% To familiarize yourself with this template, the body contains
% some examples of its use.  Look them over.  Then you can
% run LaTeX on this file.  After you have LaTeXed this file then
% you can look over the result either by printing it out with
% dvips or using xdvi.
%
% This template is based on the template for Prof. Sinclair's CS 270.

\documentclass[twoside]{article}
\usepackage{graphics}
\setlength{\oddsidemargin}{0.25 in}
\setlength{\evensidemargin}{-0.25 in}
\setlength{\topmargin}{-0.6 in}
\setlength{\textwidth}{6.5 in}
\setlength{\textheight}{8.5 in}
\setlength{\headsep}{0.75 in}
\setlength{\parindent}{0 in}
\setlength{\parskip}{0.1 in}

\usepackage{tikz}

\usepackage{amsmath,amsfonts}
\usepackage{algorithmicx}
\usepackage{algpseudocode}
\usepackage{amssymb}
\usepackage{mathtools}
\usepackage[all]{xy}
\usepackage[linesnumbered, ruled]{algorithm2e}
\SetKwRepeat{Do}{do}{while}
%
% The following commands set up the lecnum (lecture number)
% counter and make various numbering schemes work relative
% to the lecture number.
%
\newcounter{lecnum}
\renewcommand{\thepage}{\thelecnum-\arabic{page}}
\renewcommand{\thesection}{\thelecnum.\arabic{section}}
\renewcommand{\theequation}{\thelecnum.\arabic{equation}}
\renewcommand{\thefigure}{\thelecnum.\arabic{figure}}
\renewcommand{\thetable}{\thelecnum.\arabic{table}}

%
% The following macro is used to generate the header.
%
\newcommand{\lecture}[4]{
   \pagestyle{myheadings}
   \thispagestyle{plain}
   \newpage
   \setcounter{lecnum}{#1}
   \setcounter{page}{1}
   \noindent
   \begin{center}
   \framebox{
      \vbox{\vspace{2mm}
    \hbox to 6.28in { {\bf Machine Learning
                        \hfill Spring 2018} }
       \vspace{4mm}
       \hbox to 6.28in { {\Large \hfill Lecture #1: #2  \hfill} }
       \vspace{2mm}
       \hbox to 6.28in { {\it Lecturer: #3 \hfill Scribe: #4} }
      \vspace{2mm}}
   }
   \end{center}
   \markboth{Lecture #1: #2}{Lecture #1: #2}
   {\bf Disclaimer}: {\it These notes have not been subjected to the
   usual scrutiny reserved for formal publications.  They may be distributed
   outside this class only with the permission of the Instructor.}
   \vspace*{4mm}
}

%
% Convention for citations is authors' initials followed by the year.
% For example, to cite a paper by Leighton and Maggs you would type
% \cite{LM89}, and to cite a paper by Strassen you would type \cite{S69}.
% (To avoid bibliography problems, for now we redefine the \cite command.)
% Also commands that create a suitable format for the reference list.
\renewcommand{\cite}[1]{[#1]}
\def\beginrefs{\begin{list}%
        {[\arabic{equation}]}{\usecounter{equation}
         \setlength{\leftmargin}{2.0truecm}\setlength{\labelsep}{0.4truecm}%
         \setlength{\labelwidth}{1.6truecm}}}
\def\endrefs{\end{list}}
\def\bibentry#1{\item[\hbox{[#1]}]}

%Use this command for a figure; it puts a figure in wherever you want it.
%usage: \fig{NUMBER}{SPACE-IN-INCHES}{CAPTION}
\newcommand{\fig}[3]{
			\vspace{#2}
			\begin{center}
			Figure \thelecnum.#1:~#3
			\end{center}
	}
% Use these for theorems, lemmas, proofs, etc.
\newtheorem{theorem}{Theorem}[lecnum]
\newtheorem{lemma}[theorem]{Lemma}
\newtheorem{proposition}[theorem]{Proposition}
\newtheorem{claim}[theorem]{Claim}
\newtheorem{corollary}[theorem]{Corollary}
\newtheorem{definition}[theorem]{Definition}
\newenvironment{proof}{{\bf Proof:}}{\hfill\rule{2mm}{2mm}}

% **** IF YOU WANT TO DEFINE ADDITIONAL MACROS FOR YOURSELF, PUT THEM HERE:

\begin{document}
%FILL IN THE RIGHT INFO.
%\lecture{**LECTURE-NUMBER**}{**DATE**}{**LECTURER**}{**SCRIBE**}
\lecture{10}{On-Line Learning}{Liwei Wang}{Yilun Xu,Hongyin Chen}
%\footnotetext{These notes are partially based on those of Nigel Mansell.}

% **** YOUR NOTES GO HERE:

% Some general latex examples and examples making use of the
% macros follow.  
%**** IN GENERAL, BE BRIEF. LONG SCRIBE NOTES, NO MATTER HOW WELL WRITTEN,
%**** ARE NEVER READ BY ANYBODY.

% ===========================================
\section{On-line Learning with Experts Advice}
\subsection{Weighted majority algorithm}
There are $n$ experts $E_1, E_2, ..., E_n$. On each round $t = 1, 2, ..., T$,$E_i$ gives prediction $\tilde{y}_{t, i}$ . The environment(or adverary) outputs the true label $y_t$ after seeing the predictions. After receiving the true label, the algorithm then reduces the weight of each incorrect expert by a factor of $\beta \in [0, 1)$.\\
Solution:weighted majority vote and multiplicative weight updating. 
The pseudocode for the algorithm is shown below.  

Denote data classifier as $h$ and aggreator as $g$. The parameters of $h$(resp. $g$) is $\theta_h$(resp. $\theta_g$);the learning rate of $h$(resp. $g$) is $\alpha_h$(resp. $\alpha_g$).The dataset is D.
\begin{algorithm}
\caption{Adaptive Difficulty}

Initialization for the experts parameters of aggreator: $ W_{c,c'}^m = \log{\frac{\sum\limits_{i=1}^N Q(y_i=c)\mathbb{I}(y_i^m=c')}{\sum\limits_{i=1}^N Q(y_i=c)}}$(Refer  to Page 13 in our paper)\\
Initialization for $p$ using naive Bayes based on the experts parameters in aggreator.\\
Using p to initialize for the prior parameters of aggreator。\\
batch size = 64 , T = 100 \\

\For {$t = 1, 2, ..., T$} {
	Batch $(X,Y))_{64} \sim D$\\
	$left_{out} = h(X,Y)$\\
	$right_{out} = g(X,Y)$\\
	$Loss = - MIG(left_{out},right_{out},p)$(Refer  to Page 5 in our paper)\\
	
	$\theta_h = \theta_h - \alpha_{h}\nabla_{\theta_h}Loss$\\
	$\theta_g = \theta_g - \alpha_{g}\nabla_{\theta_g}Loss$\\
	$p = GetPara(g)$
	

}
\end{algorithm}

The number of mistakes made by the algorithm in T rounds:$L_T$ := $\sum^{T}_{t = 1} I[\hat{y_t} \neq y_t]$\\
The number of mistakes made by expert  $E_i$ in T rounds:$m_T^{(i)} := \sum^T_{t = 1}I[\hat{y}_{t,i} \neq y_t]$\\
	
	
\theorem{(Regret Bound)}

Fix $\beta \in [0,1)$. Let $L_T$ be the number of mistakes made by algorithm 1 after $T\ge 1$ rounds, and $m_T^{(i)}$ be the number of mistakes made by expert $E_i$.Then,the following inequality holds:

$$L_T \leqslant \frac{log \frac{1}{\beta}}{log \frac{2}{1 + \beta}} * min_{i \in [n]} m_T^{(i)} + \frac{logn}{log \frac{2}{1 + \beta}}$$\\

\begin{proof}
(Potential Function Method)


	We define our potential function as $W_t := \sum^n_{i=1}w_{t,i}$If the algorithm makes a wrong prediction at round t,this implies that\\
	$$W_{t+1} \leqslant (\frac{\beta +1}{2})W_t$$\\
	Since $W_1=n$ and $L_T$ mistakes are made after $T$ rounds,we thus have the following upper bound:
	$$W_T \leqslant  (\frac{\beta + 1}{2}) ^{L_T}*n  $$
	
	Since the weighs are non-negative and $W_T = \sum ^{n}_{i=1} w_{T,i}$,it's clear that $\forall i \in 1,2,\dots,n$,$W_T \ge w_{T,i}$.
$w_{t,i}$ updates only if i makes wrong prediction and the voting prediction is wrong on day t,so $w_{T,i} \geqslant \beta^{m_T^{(i)}}$.Combining them together, we have:
	$$\beta^{m_T^{(i)}} \le W_T \leqslant  (\frac{\beta + 1}{2}) ^{L_T}*n  $$
	$$\Rightarrow L_T \leqslant \frac{log \frac{1}{\beta}}{log \frac{2}{1 + \beta}} * min_{i \in [n]} m_T^{(i)} + \frac{logn}{log \frac{2}{1 + \beta}}$$
	

\end{proof}

\subsection{Randomized weighted majority algorithm}

In the randomized scenario of on-line learning,at each round $t \in [1,T]$,the on-line algorithm make predictions on the distribution $p_t$ over the set of experts,and receives a loss vector $l_t$,whose $i$ component $l_{t,i} \in \{0,1\}$ is the loss associated with action i.Let $L_T$ be the expected total loss incurred by the learning algorithm over T rounds, $m_{T,i}$ be the total loss of expert i.

The difference between randomized algorithm and deterministic algorithm:

1.The randomized algorithm makes randomized predictions.

2.The expert's weight is updated whenever its prediction is wrong.


The algorithm is shown below.
\begin{algorithm}
\caption{Randomized multiplicative weight updating}

initialize:  $w_{1, i} = 1,p_{1,i} =\frac{1}{N}$ for all $i \in [n]$\\
\For {$t = 1, 2, ..., T$} {
    Receive($l_t$)\\
		\For{$i = 1,2, ...,n$}{
		\If{$l_{t,i}=1$}{
		    $w_{t+1, i} \gets \beta \cdot w_{t, i} $ 
		    }
		\Else{$w_{t+1, i} \gets  w_{t, i} $ 
}
	
}
$W_{t+1}=\sum\limits_{i=1}^{N}w_{t+1,i}$\\
      \For{$i = 1,2, ...,n$}{
      $p_{t+1,i}=\frac{w_{t+1,i}}{W_{t+1}}$
      }
}
\end{algorithm}


\theorem{}
	
Fix $\beta \in [\frac{1}{2},1).$,then the loss of the randomized algorithm on any sequence can be bounded as follows:
$$L_T \leqslant (2-\beta)*min_{i \in [n]} m_T^{(i)} + \frac{logn}{1-\beta}$$

In particular,let $\beta =\max(\frac{1}{2}, 1 - \sqrt{\frac{logn}{T}})$ ,then the loss can be bounded as:
$$L_T \leqslant min_{i \in [n]}m_{T,i} + O(\sqrt{Tlogn})$$

\begin{proof}
Left as an exercise.
\end{proof}

Notice that we have $\frac{L_T}{T} \leqslant min_{i \in [n]}\frac{m_{T,i}}{T} + O(\sqrt{\frac{logn}{T}})$,fixed n ,average regret per round decreases as $O(\frac{1}{\sqrt{T}})$.

Howerver,the bound assumes that the algorithm additionallu receives T as a parameter.We can use  doubling trick to relax this requirement at the price of a small constant factor.This consists of dividing time into periods $[2^k,2^{k+1}-1]$ of length $2^k$ with $k=0,\dots,m$ and $T \ge 2^m-1$,and choose $\beta=\max(\frac{1}{2}, 1 - \sqrt{\frac{logn}{2^k}})$ in each period.We will get the same bound using doubling trick,


\subsection{Proof of minimax theorem}
\theorem{}Consider the two-person zero-sum game defined by matrix M $\in \mathbb{R}^{n \times n}$ and the scenario is similar to the randomized weighted majority algorithm.Assuming the $i^{th}$ expert is the row player of the $i^{th}$ row.Then the $i^{th}$ elements of vector $p_t$ denotes the $i^{th}$ expert's weighted on round t.We have the following on-line learning regret bound:

$$\sum^T_{t=1}p^T_tMq_t \leqslant \frac{log \frac{1}{\beta}}{1-\beta}\sum^T_{t=1}e^T_iMq_t + \frac{logn}{1-\beta},\forall i \in [n] $$
	Let $\beta = \frac{1}{1+\sqrt{\frac{2logn}{T}}}$,
	we have :$$\frac{1}{T}\sum^T_{t=1}p^T_tMq_t \leqslant \frac{1}{T} \sum^T_{t=1}l^T_iMq_t + O(\sqrt{\frac{logn}{T}})  $$

\begin{proof}
Left as an exercise.
\end{proof}

	
\theorem{Minimax theorem:}
The two-person zero-sum game defined by matrix M $\in \mathbb{R}^{n \times n}$
$$\min\limits_p\max\limits_q p^TMq=\max\limits_q\min\limits_p p^TMq$$ 
\begin{proof}
The inequality $\min\limits_p\max\limits_q p^TMq \ge \max\limits_q\min\limits_p p^TMq$ is straight forward.

To show the reverse inequality,consider an on-line learning setting where the vector $p_t$ denotes experts' weights on round t.We can assume that $q_t$ is selected in the optimal adversarial way.

To begin with,the loss of row player in one shot zero sum game can not exceed the total loss of row player after T rounds,that is $$\min\limits_p\max\limits_q p^TMq \le \max\limits_q(\frac{1}{T}\sum\limits_{t=1}^{T}p_t)^TMq \le \frac{1}{T}\sum\limits_{t=1}^{T}\max\limits_qp_t^TMq=\frac{1}{T}\sum\limits_{t=1}^{T}p_t^TMq_t$$Secondly,assume i is the best row plauer,using the bound in theorem 10.2,we have $$\frac{1}{T}\sum\limits_{t=1}^{T}p_t^TMq_t \le \frac{1}{T}\sum\limits_{t=1}^{T}e_i^TMq_t+O(\sqrt{\frac{logn}{T}})$$.

The right-hand side can be expressed and bounded as follows:
$$\frac{1}{T}\sum\limits_{t=1}^{T}p_t^TMq_t \le \frac{1}{T}\sum\limits_{t=1}^{T}e_i^TMq_t+O(\sqrt{\frac{logn}{T}}) \le e_i^TM(\frac{1}{T}\sum\limits_{t=1}^{T}q_t)+O(\sqrt{\frac{logn}{T}}) \le \max\limits_q e_i^TMq +O(\sqrt{\frac{logn}{T}})$$

So,$\forall p \succeq 0,||p||_1=1,p \in \mathbb{R}^n$,we have
$$\frac{1}{T}\sum\limits_{t=1}^{T}p_t^TMq_t=\sum\limits_{i=1}^n p_i(\frac{1}{T}\sum\limits_{t=1}^{T}p_t^TMq_t) \le \sum\limits_{i=1}^n p_i(\max\limits_q e_i^TMq+O(\sqrt{\frac{logn}{T}}))=\max\limits_q p^TMq+O(\sqrt{\frac{logn}{T}})$$
$$\Rightarrow \frac{1}{T}\sum\limits_{t=1}^{T}p_t^TMq_t \le \max\limits_q\min\limits_p p^TMq+O(\sqrt{\frac{logn}{T}})$$

Since $\lim\limits_{T \to +\infty}\sqrt{\frac{logn}{T}} =0$,we have
$$\min\limits_p\max\limits_q p^TMq \le \frac{1}{T}\sum\limits_{t=1}^{T}p_t^TMq_t \le \max\limits_q\min\limits_p p^TMq$$
\end{proof}




\section{Two papers in ICLR 2018}
\subsection{SGD learns Over-parameterized Networks that Provably Generalize on Linearly Separable data}

Content:

1)During training,optimization converges to global optima(SGD)


2)SGD finds a global optima which has good generalization ability.

3)The paper's assumptions:The architecture is a two layer network and the data is linear separable.

Can you study other networks which are more closed to practical use in this direction or weaken this paper's assumptions?

\subsection{The Implicit Bias of Gradient Descent of Separable Data}

Content:Logistic regression on linear separable data converges to the maximum margin solution

\section{Unsupervised Learning}
\subsection{Clustering}

There are n data points $x_1,x_2,\dots,x_n \in \mathbb{R}^d$,assuming the existence of  K clusters among these data.The task is to find K cluster center $c_1,c_2,\dots,c_k $ and the  objection is to minimize the sum of distance to their center,i.e $\min\limits_{\substack{c_1,\dots,c_k \\\ \text{partition of } x_1,x_2,\dots ,x_n}}\sum\limits_{i=1}^n||x_i-c_i||^2$.Although finding an exact solution to the k-means problem for arbitrary input is NP-hard, the standard approach to finding an approximate solution (often called Lloyd's algorithm or the k-means algorithm) is used widely and frequently finds reasonable solutions quickly.

\begin{algorithm}[H]
\caption{K-means Algorithm}
Initialize: stochasticly select K cluster centers $\vec c=c_1,c_2,\dots,c_k$\\
\Do{cluster center vector $\vec c$ changes}{
classify n data points by cluster centers $c_1,c_2,\dots,c_k$\\
calculate new centers vector $\vec c$
}
\Return{$\vec c$}
\end{algorithm}

  However,the approximation found can be arbitrarily bad with respect to the objective function compared to the optimal clustering.The k-means++ algorithm addresses this obstacle by specifying a procedure to initialize the cluster centers before proceeding with the standard k-means optimization iterations. With the k-means++ initialization, the algorithm is guaranteed to find a solution that is $O(\log k)$ competitive to the optimal k-means solution.(to be continued)
%\section*{References}
%\beginrefs
%\bibentry{AGM97}{\sc N.~Alon}, {\sc Z.~Galil} and {\sc O.~Margalit},
%On the Exponent of the All Pairs Shortest Path Problem,
%{\it Journal of Computer and System Sciences\/}~{\bf 54} (1997),
%pp.~255--262.

%\bibentry{F76}{\sc M. L. ~Fredman}, New Bounds on the Complexity of the 
%Shortest Path Problem, {\it SIAM Journal on Computing\/}~{\bf 5} (1976), 
%pp.~83-89.
%\endrefs


\end{document}





w